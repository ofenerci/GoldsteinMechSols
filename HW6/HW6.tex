\documentclass[10pt]{article}
\usepackage{fancyhdr}
\usepackage{amsmath}
\usepackage{amsfonts}
\usepackage{graphicx}
\pagestyle{fancy}
\headheight 35pt
\parskip 10pt \parindent 0pt 
\lhead{Classical Mechanics HW 6}
\chead{Andrei Ilyashenko}
\rhead{11-16-2010}
\cfoot{\thepage}

\begin{document}
\textbf{Chap 9 der 2}\\
We are given the transformation
\begin{align*}
  Q &= q\cos\alpha - p\sin\alpha\\
  P &= q\sin\alpha + p\cos\alpha.
\end{align*}
Lets see if this transformation satisfies the symplectic condition.  The matrix
$M$ is given by
\begin{align*}
  \left[ 
  \begin{array}[h]{c c}
    \frac{\partial Q}{\partial{q}}& \frac{\partial Q}{\partial{p}}\\
    \frac{\partial P}{\partial{q}}& \frac{\partial P}{\partial{p}}\\
  \end{array}
  \right]
  =
  \left[ 
  \begin{array}[h]{c c}
    \cos\alpha& -\sin\alpha\\
    \sin\alpha& \cos\alpha
  \end{array}.
  \right]
\end{align*}
If, we evaluate the product $MJM^{\mathrm{T}}$, we get
\begin{align*}
  \left[ 
  \begin{array}[h]{c c}
    \cos\alpha& -\sin\alpha\\
    \sin\alpha& \cos\alpha
  \end{array}
  \right]
  \left[ 
  \begin{array}[h]{c c}
    0& 1\\
    -1& 0
  \end{array}
  \right]
  \left[ 
  \begin{array}[h]{c c}
    \cos\alpha& \sin\alpha\\
    -\sin\alpha& \cos\alpha
  \end{array}
  \right]
  &=
  \left[ 
  \begin{array}[h]{c c}
    \sin\alpha& \cos\alpha\\
    -\cos\alpha& \sin\alpha
  \end{array}
  \right]
  \left[ 
  \begin{array}[h]{c c}
    \cos\alpha& \sin\alpha\\
    -\sin\alpha& \cos\alpha
  \end{array}
  \right]
  \\
  &=
  \left[ 
  \begin{array}[h]{c c}
    0& 1\\
    -1& 0
  \end{array}
  \right],
\end{align*}
thus this transformation satisfies the symplectic condition.  Now, we need to
find a generating function for this transformation.  I have decided to find
a function of the first kind.  First, I will need to solve for $p$
\begin{align*}
  Q &= q\cos\alpha-p\sin\alpha\\
  P &= q\sin\alpha+p\cos\alpha\\
  \\
  p &= \frac{q\cos\alpha-Q}{\sin\alpha}\\
  p &= \frac{P-q\sin\alpha}{\cos\alpha}.
\end{align*}
Notice that I can solve for $p$ in terms of $q$ and $Q$, or in terms of $q$ and
$P$.  The first solution is valid only if $\sin\alpha\neq0$, and the second is
valid only if $\cos\alpha\neq0$.  Since I have decided to find a generating 
function of the first kind, I will need to use the solution for $p$ in terms of
$q$ and $Q$.  This means I must restrict myself to the case where 
$\sin\alpha\neq0$, or equivalently $\alpha\neq n\pi\ n\in\mathbb{Z}$.  For these
points, the transformation simplifies to the identity transformation, whose 
generating function is well known.  Now, we can solve for the generating function
\begin{align*}
  -\frac{\partial F_1}{\partial Q} &= P\\
  -\frac{\partial F_1}{\partial Q} &= q\sin\alpha+p\cos\alpha\\
  -\frac{\partial F_1}{\partial Q} &= q\sin\alpha+\frac{q\cos\alpha-Q}{\sin\alpha}\cos\alpha\\
  F_1 &= -Qq\sin\alpha+\left( \frac{1}{2}Q^2-Qq\cos\alpha \right)\cot\alpha+g(q).
\end{align*}
However, we also require that
\begin{align*}
  \frac{\partial F_1}{\partial q} &= p\\
  -Q\sin\alpha+\left( -Q\cos\alpha \right)\cot\alpha+g'(q) &=  \frac{q\cos\alpha-Q}{\sin\alpha}\\
  -Q\sin^2\alpha+ -Q\cos^2\alpha+\sin\alpha g'(q) &=  q\cos\alpha-Q\\
  \sin\alpha g'(q) &=  q\cos\alpha\\
  g(q) &=  \frac{1}{2}q^2\cot\alpha.
\end{align*}
So, the generating function is
\begin{align*}
  F_1 &= -Qq\sin\alpha+\left( \frac{1}{2}Q^2+\frac{1}{2}q^2-Qq\cos\alpha \right)\cot\alpha+g(q).
\end{align*}
The physical significance of the transformation at $\alpha=0$, is the identity transformation,
and the physical significance of the transformation at $\alpha=\frac{\pi}{2}$ is that coordinate
is exchanged with the negative of momentum, and momentum is exchanged with coordinate.  In the
first case, the generating function fails, this is because a generating function of the first 
kind assumes that $q$ and $Q$ are independent, which is not the case if $q=Q$.\\
\textbf{Chap 9 der 10}\\
The transformation is given by
\begin{align*}
  Q &= \frac{\alpha p}{x}\\
  P &= \beta x^2.
\end{align*}
In order for this transformation to be canonical, we must have
\begin{align*}
  [Q,P]_{qp} &= 1\\
  \frac{\partial Q}{\partial q}\frac{\partial P}{\partial p} - \frac{\partial Q}{\partial p}\frac{\partial P}{\partial q} &= 1\\
  -2\alpha\beta &= 1\\
  \beta &= -\frac{1}{2\alpha}.
\end{align*}
Thus, the transformation is only canonical if $\beta = -\frac{1}{2\alpha}$.  A generating function of the first
kind for this transformation is
\begin{align*}
  F_1 &= \frac{Qx^2}{2\alpha}.
\end{align*}
This is easy to verify
\begin{align*}
  \frac{\partial F_1}{\partial x} &= \frac{Qx}{\alpha} = p\\
  -\frac{\partial F_1}{\partial Q} &= -\frac{x^2}{\alpha} = P.
\end{align*}
The Hamiltonian in the new variables is
\begin{align*}
  H &= \frac{p^2}{2m}+\frac{kx^2}{2}=-\frac{Q^2P}{m\alpha}-Pk\alpha,
\end{align*}
and the equations of motion are
\begin{align*}
  \dot{P} &= -\frac{\partial H}{\partial Q}\\
  \dot{Q} &= \frac{\partial H}{\partial P}\\
  \dot{P} &= \frac{2QP}{m\alpha}\\
  \dot{Q} &= -\frac{Q^2}{m\alpha}-k\alpha.
\end{align*}
\textbf{Chap 9 ex 22}\\
In terms of Poisson brackets, the canonical condition is the following system 
of equations
\begin{align*}
  \frac{\partial P_1}{\partial p_1} + \frac{\partial P_1}{\partial p_2} &= 0\\
  2q_1\frac{\partial P_2}{\partial p_1} &= 0\\
  2q_1\frac{\partial P_1}{\partial p_1} &= 1\\
  q_2\frac{\partial P_2}{\partial p_1} + q_1\frac{\partial P_2}{\partial p_2} &= 1.
\end{align*}
One of the poisson brackets is zero for any choice of $P$, and the last poisson
bracket I will deal with later.  Here is the most general form for $P_1$ and
$P_2$
\begin{align*}
  P_1 &= \frac{p_1-p_2}{2q_1}+f_1(q_1,q_2)\\
  P_2 &= p_2+f_2(q_1,q_2).
\end{align*}
Now, a good choice for $P_1$ and $P_2$ is
\begin{align*}
  P_1 &= \frac{p_1-p_2}{2q_1}\\
  P_2 &= p_2+(q_1+q_2)^2.
\end{align*}
The hamiltonian can be written entirely in terms of $P_1$ and $P_2$, but lets
see if this transformation is canonical.  The last poisson bracket is
\begin{align*}
  \frac{\partial P_1}{\partial p_1}\frac{\partial P_2}{\partial q_1} + \frac{\partial P_1}{\partial p_2}\frac{\partial P_2}{\partial q_2} - 
  \frac{\partial P_1}{\partial q_1}\frac{\partial P_2}{\partial p_1} - \frac{\partial P_1}{\partial q_2}\frac{\partial P_2}{\partial p_2}  
  &= 0\\
  \frac{partial f_1}{\partial q_1}-\left( \frac{\partial f_2}{\partial q_1}\frac{1}{2q_1} - \frac{\partial f_2}{\partial q_2}\frac{1}{2q_1} \right) &= 0.
\end{align*}
So, if $f_1=0$ and $f_2=(q_1+q_2)^2$, then this condition will be satisfied so the transformation will be canonical.
The new hamiltonian is
\begin{align*}
  H &= \left( \frac{p_1-p_2}{2q_1} \right)^2+p_2+(q_1+q_2)^2\\
    &= (P_1)^2+P_2.
\end{align*}
So, the equations of motion are
\begin{align*}
  P_1 &= \alpha\\
  P_2 &= \beta\\
  Q_1 &= 2\alpha t + c_1\\
  Q_2 &= t + c_2.
\end{align*}
With a little bit more algebra, these equations can be put in terms of 
$q_1,q_2$, and $p_1,p_2$ using the inverse of the original canonical 
transformation.  In this for the solution will probably not be linear,
so by doing a transformation we essentially solved the problem without
having to solve the original Hamilton's equations.

\textbf{Chap 9 ex 24}\\
\textbf{a)}\\
I will show that this transformation is canonical using Poisson brackets
\begin{align*}
  [Q,P]_{qp} &= \frac{\partial Q}{\partial q}\frac{\partial P}{\partial p}-\frac{\partial Q}{\partial p}\frac{\partial P}{\partial q}\\
  [Q,P]_{qp} &= \frac{i\alpha}{2i\alpha}+\frac{1}{2}=1.
\end{align*}
Thus, the transformation is canonical.  A generating function of the first kind
is given by
\begin{align*}
  F_1 &= -\frac{Q^2}{4i\alpha}+qQ-\frac{i\alpha}{2}q^2.
\end{align*}
This can be easily verified
\begin{align*}
  -\frac{\partial F_1}{\partial Q} &= \frac{Q}{2i\alpha}-q\\
  &=\frac{Q-2qi\alpha}{2i\alpha}=\frac{p-qi\alpha}{2i\alpha}=P\\
  \\
  \frac{\partial F_1}{\partial q} &= Q-i\alpha q=p.
\end{align*}
\textbf{b)}\\
In terms of there new variables, the solution to the harmonic oscillator is given by
\begin{align*}
  \dot{Q} &= -kq+i\alpha\frac{p}{m} = \frac{Q}{m}-P\frac{(1+k)2i\alpha}{(1-i\alpha)m}\\
  \dot{P} &= \frac{-kq}{2i\alpha}-\frac{p}{2m} = \frac{-Q}{2mi\alpha}-P\frac{(k-1)}{m(1-i\alpha)}
\end{align*}
\textbf{Chap 9 ex 25}\\
\textbf{a)}\\
The equations of motion are given by: 
\begin{align*}
  \dot{p} &= -\frac{\partial H}{\partial q}\\
  \dot{q} &= \frac{\partial H}{\partial p}\\
  \\
  \dot{p} &= \frac{1}{2q}+2p^2q^3\\
  \dot{q} &= pq^4.
\end{align*}
\textbf{b)}\\
The following transformation:
\begin{align*}
  q&=\frac{1}{Q\sqrt{k}}\\
  p&=\frac{PQ^2k}{\sqrt{m}},
\end{align*}
will make the Hamiltonian in part a reduce to a harmonic oscillator.  However,
this transformation is not canonical, since $[Q,P]=\sqrt{\frac{k}{m}}$.  The 
only canonical transformation I could find was
\begin{align*}
  q&=\frac{\sqrt{m}}{P}\\
  p&=\frac{QP^2}{\sqrt{m}},
\end{align*}
but this reduces to the special harmonic oscillator with $k=m$.  For this 
transformation, the equation of motion from part a is 
\begin{align*}
  \dot{q} &= pq^4\\
  -\dot{P}\frac{\sqrt{m}}{P^2} &= \frac{m^2}{P^2} \frac{Q}{\sqrt{m}}\\
  \dot{P} &= -mQ,
\end{align*}
which is just the equation of motion for the harmonic oscillator with $k=m$.
So, the equation from part $a$ is satisfied.

\end{document}
