\documentclass[10pt]{article}
\usepackage{fancyhdr}
\usepackage{amsmath}
\usepackage{amsfonts}
\usepackage{graphicx}
\pagestyle{fancy}
\headheight 35pt
\parskip 10pt \parindent 0pt 
\lhead{Classical Mechanics HW 7}
\chead{Andrei Ilyashenko}
\rhead{11-22-2010}
\cfoot{\thepage}

\begin{document}
\textbf{Chap 10 der 1}\\
We want the new Hamilton, $K(Q,P,t)$, to be a function of coordinate only, so
\begin{align*}
  \frac{\partial K}{\partial P} &= 0.
\end{align*}
Hamilton's equations tell us that
\begin{align*}
  \frac{\partial K}{\partial P} &= \dot{Q}.
\end{align*}
So,
\begin{align*}
  \dot{Q} &= 0\\
  Q &= \textrm{Constant}.
\end{align*}
If we use a generating function of the second kind (others may work as well?),
then we have
\begin{align*}
  Q &= \frac{\partial F_2}{\partial P},
\end{align*}
remember that $Q$ is a constant this would imply
\begin{align*}
  F_2(q,P,t) &= QP + f(q,t).
\end{align*}
The other transformation equation gives us
\begin{align*}
  p &= \frac{\partial F_2}{\partial q}\\
  p &= \frac{\partial f}{\partial q}.
\end{align*}
I want to apply Hamilton's equations to this, but they involve $\dot p$, so
we need to take a time derivative of both sides
\begin{align*}
  \dot p &= \frac{d}{dt}\frac{\partial f}{\partial q}\\
  -\frac{\partial H}{\partial q} &= \frac{d}{dt}\frac{\partial f}{\partial q}.
\end{align*}
Does this statement imply
\begin{align*}
  \frac{d}{dt}f &= -H\textrm{?}
\end{align*}
In any case, if $f$ satisfies the above differential equation, then we will
have a transformation function, $F_2=QP+f(q,t)$, where $Q$ is a constant that
generates a transformation where the new Hamiltonian depends only on
coordinate.\\
\textbf{Chap 10 ex. 5}\\
Show that the function
\begin{align*}
  S &= \frac{m\omega}{2}\left( q^2+\alpha^2 \right)\cot\omega t - m\omega q\alpha\csc\omega t,
\end{align*}
is a solution of the Hamilton-Jacobi for Hamilton's principal function for the
linear harmonic oscillator with
\begin{align*}
  H &= \frac{1}{2m}\left( p^2+m^2\omega^2q^2 \right).
\end{align*}
Show that this function generates a correct solution to the motion of the 
harmonic oscillator.\\
\textbf{Chap 10 ex. 6}\\
A charged particle is constrained to move in a plane under the influence of
a central force potential (nonelectromagnetic) $V=\frac{1}{2}kr^2$, and a 
constant magnetic $B$ perpendicular to the plane, so that
\begin{align*}
  A &= \frac{1}{2}B\times r.
\end{align*}
Set up the Hamilton-Jacobi equation for Hamilton's characteristic function
in plane polar coordinates.  Separate the equation and reduce it to
quadratures.  Discuss the motion if the canonical momentum $p_{\theta}$ is
zero at time $t=0$.
\textbf{Chap 10 ex. 8}\\
Suppose the potential in a problem of one degree of freedom is linearly
dependent upon time, such that the Hamiltonian has the form
\begin{align*}
  H &= \frac{p^2}{2m}-mAtx,
\end{align*}
where $A$ is a constant.  Solve the dynamical problem by means of
Hamilton's principal function, under the initial conditions $t=0$, $x=0$,
$p=mx_0$.
\textbf{Chap 10 ex. 11}\\
A particle is constrained to move on a roller coaster, the equation whose
curve is 
\begin{align*}
  z &= A\cos^2\frac{2\pi x}{\lambda}.
\end{align*}
There is the usual constant downward force of gravity. Discuss the system
trajectories in phase space under all possible initial conditions,
describing the phase space orbits in as much detail as you can, paying
special attention to turning points and transitions between different types
of motion.
\end{document}

