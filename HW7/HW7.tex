\documentclass[10pt]{article}
\usepackage{fancyhdr}
\usepackage{amsmath}
\usepackage{amsfonts}
\usepackage{graphicx}
\pagestyle{fancy}
\headheight 35pt
\parskip 10pt \parindent 0pt 
\lhead{Classical Mechanics HW 7}
\chead{Andrei Ilyashenko}
\rhead{11-22-2010}
\cfoot{\thepage}

\begin{document}
\textbf{Chap 9 der 15}\\
\textbf{a)}\\
The transformation is
\begin{align*}
  Q &= q^{\alpha}\cos{\beta p}\\
  P &= q^{\alpha}\sin{\beta p}.
\end{align*}
In order to be canonical, it must satisfy
\begin{align*}
  [Q,P] &= 1\\
  \frac{\partial Q}{\partial q}\frac{\partial P}{\partial p} - \frac{\partial Q}{\partial p}\frac{\partial P}{\partial q} &= 1\\
  \alpha\beta q^{2\alpha-1}\cos^2{\beta p} + \alpha\beta q^{2\alpha-1}\sin^2{\beta p} &= 1\\
  \alpha\beta q^{2\alpha-1} &= 1\\
  \frac{1}{\alpha\beta} &= q^{2\alpha-1}.
\end{align*}
This equation will be satisfied for all $q$ only when $2\alpha-1=0$ so, we 
must have $\alpha=1/2$ $\beta=2$.\\
\textbf{b)}\\
If this could be an extended canonical transformation, then we would have
\begin{align*}
  [Q,P] &= \lambda\\
  \alpha\beta q^{2\alpha-1}\cos^2{\beta p} + \alpha\beta q^{2\alpha-1}\sin^2{\beta p} &= \lambda\\
  \frac{\lambda}{\alpha\beta} &= q^{2\alpha-1}.
\end{align*}
Again $\alpha=1/2$, but this time $\beta = 2\lambda$.  The transformation function
of the third kind, $F_3(Q,p)$ for this transformation must satisfy
\begin{align*}
  -\frac{\partial F_3}{\partial p} &= \lambda q\\
  -\frac{\partial F_3}{\partial Q} &= P\\
  \\
  -\frac{\partial F_3}{\partial p} &= \lambda\frac{Q^2}{\cos^2(\beta p)}\\
  -\frac{\partial F_3}{\partial Q} &= Q\tan{\beta p}.
\end{align*}
The transformation function satisfying these conditions is given by
\begin{align*}
  F_3(Q,p) &= -\frac{1}{2\beta}Q^2\tan(\beta p).
\end{align*}
\textbf{c)}\\
Well, the factor of $\lambda$ appears because when we take a derivative with
respect to $p$, we get a factor of $\beta=2\lambda$.  If the transformation 
function was instead
\begin{align*}
  F_3(Q,p) &= -\frac{1}{2\beta}Q^2\tan(\beta p),
\end{align*}
then when we take the derivative the $\beta$ in the denominator would cancel
the $\beta$ we get from differentiating.  So, the new equations would be
\begin{align*}
  -\frac{\partial F_3}{\partial p} &= q\\
  -\frac{\partial F_3}{\partial Q} &= P\\
  \\
  -\frac{\partial F_3}{\partial p} &= \frac{Q^2}{\cos^2(\beta p)}\\
  -\frac{\partial F_3}{\partial Q} &= Q\frac{1}{\beta}\tan{\beta p}.
\end{align*}
Consequently, the modified transformation equations are
\begin{align*}
  Q &= q^{\alpha}\cos(\beta p)\\
  P &= q^{\alpha}\sin(\beta p)\frac{1}{\beta}.
\end{align*}
\textbf{Chap 9 Ex 28}\\
\textbf{a)}\\
First, we need to express the velocities in terms of the conjugate momentum 
and coordinate.
Assuming this charged particle is not relativistic, the Lagrangian for a charged
particle moving in a magnetic field is
\begin{align*}
  \mathcal{L} &= \frac{1}{2}m\dot{q_i}\dot{q_i} + e\dot{q_i}A_i\\
\end{align*}
So, the conjugate momentum is given by
\begin{align*}
  p_i &= \frac{\partial L}{\partial \dot{q_i}}\\
      &= m\dot{q_i}+eA_i.
\end{align*}
In the case of a uniform magnetic field of magnitude $B$ pointing in the $z$,
or third dimension, this becomes
\begin{align*}
  \mathcal{L} &= \frac{1}{2}m\dot{q_i}\dot{q_i} - eB\dot{q_1}q_2 + eB\dot{q_2}q_1\\
  p_1 &= m\dot{q_1}-eBq_2\\
  p_2 &= m\dot{q_2}+eBq_1\\
  p_3 &= m\dot{q_3}.
\end{align*}
So, the velocities in terms of $q$ and $p$ are
\begin{align*}
  v_1 &= \frac{p_1+eBq_2}{2m}\\
  v_2 &= \frac{p_2-eBq_1}{2m}\\
  v_3 &= \frac{p_3}{2m}.
\end{align*}
Thus, the Poisson brackets are
\begin{align*}
  [v_i,v_j] &= \frac{\partial v_i}{\partial q_k}\frac{\partial v_j}{\partial p_k} - \frac{\partial v_j}{\partial q_k}\frac{\partial v_i}{\partial p_k}\\
  [v_1,v_2] &= \frac{eB+eB}{2m}=\frac{eB}{m}\\
  [v_1,v_3] &= 0\\
  [v_2,v_3] &= 0.
\end{align*}
\textbf{b)}\\
Since, we already have $v_i$ in terms of $p$ and $q$, evaluating these
Poisson brackets is straight forward
\begin{align*}
  [q_i,v_j] &= \frac{\partial v_j}{\partial p_i} = \frac{1}{2m}\delta_{ij}\\
  \\
  [p_i,v_j] &= -\frac{\partial v_j}{\partial q_i}\\
  [p_2,v_1] &= \frac{eB}{2m}\\
  [p_1,v_2] &= -\frac{eB}{2m}\\
  \\
  [q_i,\dot{p_j}] &= -[v_i,p_j] = [p_j,v_i]\\
  [q_2,\dot{p_1}] &= -\frac{eB}{2m}\\
  [q_1,\dot{p_2}] &= \frac{eB}{2m}\\
  \\
  [p_i,\dot{p_j}] &= [p_i,\frac{\mathcal{L}}{q_j}]\\
  [p_1,\dot{p_1}] &= eB[p_1,v_2] = -\frac{e^2B^2}{2m}\\
  [p_1,\dot{p_2}] &= -eB[p_2,v_1] = -\frac{e^2B^2}{2m}.
\end{align*}
Poisson brackets equal to 0 have not been written out explicitly.
\textbf{Chap 9 Ex 31}\\
The Hamiltonian for the 1D harmonic oscillator is
\begin{align*}
  \mathcal{H} &= \frac{p^2}{2m}+\frac{1}{2}m\omega^2q^2.
\end{align*}
If $u(q,p,t)=ln(p+im\omega q)-i\omega t$ is a constant of motion, then we must have
\begin{align*}
  [u,\mathcal{H}] &= -\frac{\partial u}{\partial t}\\
  \frac{ip\omega-m\omega^2q}{p+im\omega q} &= i\omega\\
  ip\omega-m\omega^2q &= i\omega p-m\omega^2 q.
\end{align*}
Thus, it is a constant of motion.  The physical significance probably has 
something to do with the symmetry of the harmonic oscillator.
\textbf{Chap 9 Ex 32}\\
The Hamiltonian we are given is
\begin{align*}
  \mathcal{H} &= q_1p_1-q_2p_2-aq_1^2+bq_2^2.
\end{align*}
First, lets check if $F_1=\frac{p_1-aq_1}{q_2}$ is a constant of motion as in
the last problem
\begin{align*}
  [F_1,\mathcal{H}] &= -\frac{\partial F_1}{\partial t}\\
  \frac{-a}{q_2}(q_1) + \frac{p_1-aq_1}{q_2} - \left( \frac{p_1-2aq_1}{q_2} \right) &= 0\\
  \frac{-2aq_1}{q_2} + \frac{2aq_1}{q_2} + \frac{p_1}{q_2} - \frac{p_1}{q_2}  &= 0\\
  0 &= 0\\
\end{align*}
Thus, $F_1$ is a constant of motion.  Now, for $F_2=q_1q_2$
\begin{align*}
  [F_2,\mathcal{H}] &= -\frac{\partial F_2}{\partial t}\\
  q_1q_2 - q_1q_2 &= 0\\
\end{align*}
So, $F_2$ is also a constant of motion.  Assuming a third constant of motion $F_3$, can 
somehow be generated using Jacobi's Identity, lets see what happens when we simplify it
\begin{align*}
  [F_1,[F_2,\mathcal{H}]] + [\mathcal{H},[F_1,F_2]] + [F_2,[\mathcal{H},F_1]] &= 0\\
   [\mathcal{H},[F_1,F_2]]  &= 0\\
   [\mathcal{H},F_3]  &= 0\\
   [F_3,\mathcal{H}]  &= 0
\end{align*}
So, $F_3=[F_1,F_2]$.

\textbf{Chap 9 Ex 36}\\
\textbf{a)}\\
Using the theorem concerning Poisson brackets of vector functions and components
of the angular momentum, show that if \textbf{F} and \textbf{G} are two vector
functions of the coordinates and momenta only, then
\begin{align*}
  [F\cdot L,G\cdot L] &= L\cdot(G\times F)+L_iL_j[F_i,G_j]\\
\end{align*}
So, lets start with the left hand side and show that it becomes the right hand
side
\begin{align*}
  [F\cdot L,G\cdot L] &= [F_iL_i,G_jL_j]\\
                      &= F_i[L_i,G_jL_j] + L_i[F_i,G_jL_j]\\
                      &= F_iL_j[L_i,G_j] + F_iG_j[L_i,L_j] + L_iG_j[F_i,L_j] + L_iL_j[F_i,G_j]\\
                      &= L_iL_j[F_i,G_j] + F_iL_j[L_i,G_j] + F_iG_j[L_i,L_j] + L_iG_j[F_i,L_j]\\
                      &= L_iL_j[F_i,G_j] - F_iL_j[G_j,L_i] + F_iG_j[L_i,L_j] + L_iG_j[F_i,L_j]\\
                      &= L_iL_j[F_i,G_j] - F_iL_j\epsilon_{jik}G_k + F_iG_j\epsilon{ijk}L_k + L_iG_j\epsilon_{ijk}F_k\\
                      &= L_iL_j[F_i,G_j] - L\cdot(F\times G) + L\cdot(F\times G) + L_i(G\times F)\\
                      &= L_iL_j[F_i,G_j] + L_i(G\times F)\\
                      &= L_i(G\times F) + L_iL_j[F_i,G_j]\\
\end{align*}
\textbf{b)}\\
Let $F$ be the unit vector in the $\mu$ direction, $G$ be the unit vector in the $\nu$ direction, and $\lambda$ by the third direction (so if $\mu=1$ and $\nu=2$ then $\lambda$ will equal 3.  Ok,
now if we substitute the value of $F$ and $G$ into the equation from part a we get
\begin{align*}
  [F\cdot L,G\cdot L] &= L\cdot(G\times F)+L_iL_j[F_i,G_j]\\
  [F_iL_i,G_jL_j]     &= L_i\epsilon_{ijk}G_jF_k+L_iL_j[F_i,G_j].
\end{align*}
Now, we know that the only non-zero terms of $F_k$ and $G_j$ are $F_{\mu}$ and $G_{\nu}$ so the equation above becomes
\begin{align*}
  [F_iL_i,G_jL_j] &= L_i\epsilon_{ijk}G_jF_k+L_iL_j[F_i,G_j]\\
  [F_{\mu}L_{\mu},G_{\nu}L_{\nu}] &= L_i\epsilon_{i\nu\mu}G_{\nu}F_{\mu}+L_{\mu}L_{\nu}[F_{\mu},G_{\nu}]\\
\end{align*}
remember that $F_{\mu}=G_{\nu}=1$ since these are unit vectors, so the expression above becomes
further simplified
\begin{align*}
  [F_{\mu}L_{\mu},G_{\nu}L_{\nu}] &= L_i\epsilon_{i\nu\mu}G_{\nu}F_{\mu}+L_{\mu}L_{\nu}[F_{\mu},G_{\nu}]\\
  [L_{\mu},L_{\nu}] &= L_i\epsilon_{i\nu\mu}+L_{\mu}L_{\nu}[1,1]\\
  [L_{\mu},L_{\nu}] &= L_i\epsilon_{i\nu\mu}.
\end{align*}
Lets call the direction perpendicular to $\mu$ and $\nu$ as $\lambda$, so the equation above
becomes
\begin{align*}
  [L_{\mu},L_{\nu}] &= L_{\lambda}\epsilon_{\lambda\nu\mu}\\
  [L_{\mu},L_{\nu}] &= I_{\lambda}\omega_{\lambda}\epsilon_{\lambda\nu\mu}.
\end{align*}
Therefore all of the sums reduce to simply one expression, where $\lambda$ is just equal to whatever $\mu$ and $\nu$ is not equal.  In the case where $\mu=\nu$ it is unclear what to pick for $\lambda$, but in this case the epsilon is equal to zero since $\mu=\nu$ so it doesn't matter. 
\textbf{c)}\\
So, the equation of motion for $L_i$ is
\begin{align*}
  \frac{dL_i}{dt} &= [L_i,H] + \frac{\partial L_i}{\partial t}\\
  \frac{dL_i}{dt} - [L_i,H] &= \frac{\partial L_i}{\partial t}\\
  \frac{dL_i}{dt} - \epsilon_{ikj}\omega_j\omega_kI_k &= N_i\\
  \frac{dL_i}{dt} + \epsilon_{ijk}\omega_j\omega_kI_k &= N_i.
\end{align*}

\end{document}

